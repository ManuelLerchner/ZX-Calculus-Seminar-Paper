\documentclass[conference]{IEEEtran}
\IEEEoverridecommandlockouts
% The preceding line is only needed to identify funding in the first footnote. If that is unneeded, please comment it out.
\usepackage{cite}
\usepackage{amsmath,amssymb,amsfonts}
\usepackage{algorithmic}
\usepackage{graphicx}
\usepackage{textcomp}
\usepackage{xcolor}
\usepackage{tikz}
\usepackage{subfiles}
\usepackage{url}
\usepackage{pdfpages}

\usetikzlibrary{quantikz}
\usetikzlibrary{zx-calculus}


\def\BibTeX{{\rm B\kern-.05em{\sc i\kern-.025em b}\kern-.08em
    T\kern-.1667em\lower.7ex\hbox{E}\kern-.125emX}}

\begin{document}

\subfile{sections/titlepage}

\tableofcontents

\title{ZX-Calculus}

\author{
    \IEEEauthorblockN{ Manuel Lerchner}
    \IEEEauthorblockA{
        \textit{Technical University of Munich}\\
        Munich, Germany}
}

\maketitle

\begin{abstract}
    ZX-Calculus is a graphical language which extends classical quantum circuits by splitting up logic gates into even smaller building blocks. Those building blocks are called spiders and are represented by colored nodes in a graph. Together with edges connecting those nodes, they form a ZX-diagram. Using a set of rewrite rules, ZX-diagrams can be transformed into each other. This allows for a more intuitive way of reasoning about the optimization of quantum circuits, as there are fewer rewrite rules to remember than in the classical logic gate model. In this paper, I will introduce the ZX-Calculus and its rewrite rules, as well as some of its applications. I will give a special focus on the optimization of quantum circuits, as this is one of the main applications of the ZX-Calculus.
\end{abstract}

\begin{IEEEkeywords}
    quantum computing, ZX-Calculus, quantum circuits, circuit optimization
\end{IEEEkeywords}


\subfile{sections/introduction.tex}

\subfile{sections/introduction_to_zx}


\section{Motivation}


\section{Problem Statement}

\section{Solution}

\section{Evaluation}

\section{Results}

\section{Future Work}

\section{Conclusion}



\bibliographystyle{plain}
\bibliography{refs}



\end{document}
