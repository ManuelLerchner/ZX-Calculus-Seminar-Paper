\section{Introduction to ZX-Calculus}

ZX-Calculus has been kickstarted by Coecke and Duncan in 2008 \cite{Coecke2007graphicalcalculus}. Since then, it has been used to prove a lot of interesting results in quantum computing. It was initially used in the field of Measurement Based Quantum Computing (MBQC) \cite{duncan2012graphical}. But recently it has found wide application in quantum circuit optimization and verification  \cite{vandewetering2020zxcalculus}.

\subsection{ZX-Diagrams}

A ZX-diagram is a graph consisting of nodes and edges. The nodes are called spiders and are colored either green or red. Additionally, they may carry a phase value $\alpha \in [0, 2\pi)$. Spiders are visuallized in the following way:


\vspace{5pt}
\begin{ZX}
    \leftManyDots{n}  \zxZ{\alpha} \rightManyDots{m}
\end{ZX}

\vspace{5pt}


\begin{ZX}
    \leftManyDots{n}  \zxX{\alpha} \rightManyDots{m}
\end{ZX}
\vspace{5pt}


Here $n$ and $m$ are the number of incoming and outgoing edges, respectively. The phase value $\alpha$ is optional and can be omitted if it is zero.

It is important to remember that each spider represents a linear map from $n$ to $m$ qubits. The values of those linear maps can be directly calculated using the following formulas:

\vspace{5pt}
\zx{\leftManyDots{n}  \zxX{\alpha} \rightManyDots{m}} $\equiv |\underbrace{0\dots 0}_{m}\rangle \langle \underbrace{0\dots0}_{n}| + e^{i\alpha}|\underbrace{1\dots1}_{m}\rangle \langle \underbrace{1\dots1}_{n}|$

\vspace{5pt}
\begin{ZX}
    \leftManyDots{n}  \zxZ{\alpha} \rightManyDots{m}
\end{ZX} $\equiv |\underbrace{+\dots +}_{m}\rangle \langle \underbrace{+\dots+}_{n}| - e^{i\alpha}|\underbrace{-\dots-}_{m}\rangle \langle \underbrace{-\dots-}_{n}|$
\vspace{5pt}



\subsection{Example Spiders and their Linear Maps}

By just using single spiders, we can already represent a lot of quantum gates. The most important ones are shown in figure \ref{fig:spiders}. Note that we are going to ignore the global scalar value of a circuit, as it does not change the outcome of the simplificiation process. It is always possible to add it back at the end.

\subsection*{Verifying some Unitary Gates}

Let's verify that the Pauli-Z gate and the Pauli-X gate can indeed be represented by the spiders shown in figure \ref{fig:spiders}.


\begin{align}
    \begin{ZX}[ampersand replacement=\&]
        \zxN{} \rar \&[\zxwCol] \zxZ{\pi} \rar \&[\zxwCol] \zxN{}
    \end{ZX}
     & \equiv |\underbrace{0\dots 0}_{1}\rangle \langle \underbrace{0\dots0}_{1}| + e^{i\pi}|\underbrace{1\dots1}_{1}\rangle \langle \underbrace{1\dots1}_{1}| \\
     & \equiv
    |0\rangle \langle 0| - |1\rangle \langle 1|                                                                                                                \\
     & \equiv
    \begin{pmatrix}
        1 \\
        0 \\
    \end{pmatrix}
    \begin{pmatrix}
        1 & 0 \\
    \end{pmatrix}
    - \begin{pmatrix}
          0 \\
          1 \\
      \end{pmatrix}
    \begin{pmatrix}
        0 & 1 \\
    \end{pmatrix}                                                                                                                                             \\
     & \equiv
    \begin{pmatrix}
        1 & 0  \\
        0 & -1 \\
    \end{pmatrix}                                                                                                                                             \\
     & \equiv Z
\end{align}


\begin{figure}
    \includegraphics[width=\linewidth]{images/single_spider_unitaries.png}
    \caption{Spiders representing quantum gates
            {\cite{vandewetering2020zxcalculus}[P.87]}}
    \label{fig:spiders}
\end{figure}






