\begin{abstract}
    ZX-Calculus is a graphical language that extends classical quantum circuits by splitting up logic gates into even smaller building blocks. Those building blocks are called spiders and are represented by colored nodes in a graph. Together with edges connecting those nodes, they form a ZX diagram. A set of rewrite rules allows us to perform transformations on those ZX diagrams. ZX-Calculus promises a more intuitive way of reasoning about optimizing quantum circuits, as there are fewer rewrite rules to remember than in the classical logic gate model. This paper will introduce the ZX-Calculus, its rewrite rules, and some of its applications. I will give a particular focus on optimizing quantum circuits, as this is one of the main applications of the ZX-Calculus.
\end{abstract}

\begin{IEEEkeywords}
    quantum computing, ZX-Calculus, quantum circuits, circuit optimization
\end{IEEEkeywords}