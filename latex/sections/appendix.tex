\section{Appendix}

\subsection{ZX-Representation of the Pauli Z-Gate}
\label{appendix:pauli-z-gate-as-spider}

\begin{figure}[h]
    \begin{align*}
        \begin{ZX}
            \zxN{} \rar &[\zxwCol] \zxZ{\pi} \rar &[\zxwCol] \zxN{}
        \end{ZX}
         & \equiv |\underbrace{0\dots 0}_{1}\rangle \langle \underbrace{0\dots0}_{1}| + e^{i\pi}|\underbrace{1\dots1}_{1}\rangle \langle \underbrace{1\dots1}_{1}| \\
         & =
        |0\rangle \langle 0| - |1\rangle \langle 1|                                                                                                                \\
         & =
        \begin{pmatrix}
            1 \\
            0 \\
        \end{pmatrix}
        \begin{pmatrix}
            1 & 0 \\
        \end{pmatrix}
        - \begin{pmatrix}
              0 \\
              1 \\
          \end{pmatrix}
        \begin{pmatrix}
            0 & 1 \\
        \end{pmatrix}                                                                                                                                             \\
         & =
        \begin{pmatrix}
            1 & 0  \\
            0 & -1 \\
        \end{pmatrix}                                                                                                                                             \\
         & = Z
        \\
    \end{align*}
    \caption{Pauli-Z gate represented as a ZX-Diagram}
\end{figure}

\subsection{ZX-Representation of the Pauli X-Gate}
\label{appendix:pauli-x-gate-as-spider}


\begin{figure}[h]
    \begin{align*}
        \begin{ZX}
            \zxN{} \rar &[\zxwCol] \zxX{\pi} \rar &[\zxwCol] \zxN{}
        \end{ZX}
         & \equiv |\underbrace{+\dots +}_{1}\rangle \langle \underbrace{+\dots +}_{1}| + e^{i\pi}|\underbrace{-\dots -}_{1}\rangle \langle \underbrace{-\dots -}_{1}| \\
         & =
        |+\rangle \langle +| - |-\rangle \langle -|                                                                                                                   \\
         & =
        \frac{1}{\sqrt{2}}
        \begin{pmatrix}
            1 \\
            1 \\
        \end{pmatrix}
        \frac{1}{\sqrt{2}}
        \begin{pmatrix}
            1 & 1 \\
        \end{pmatrix}
        -
        \frac{1}{\sqrt{2}}
        \begin{pmatrix}
            1  \\
            -1 \\
        \end{pmatrix}
        \frac{1}{\sqrt{2}}
        \begin{pmatrix}
            1 & -1 \\
        \end{pmatrix}                                                                                                                                                \\
         & =
        \frac{1}{2}
        \begin{pmatrix}
            1 & 1 \\
            1 & 1 \\
        \end{pmatrix}
        -
        \frac{1}{2}
        \begin{pmatrix}
            1  & -1 \\
            -1 & 1  \\
        \end{pmatrix}                                                                                                                                                \\
         & =
        \begin{pmatrix}
            0 & 1 \\
            1 & 0 \\
        \end{pmatrix}                                                                                                                                                \\
         & = X
    \end{align*}
    \caption{Pauli-X gate represented as a ZX-Diagram}
\end{figure}

\subsection{ZX-Representation of the Pauli Y-Gate}
\label{appendix:pauli-y-gate-as-spider}

\begin{figure}[h]
    \begin{align*}
        \begin{ZX}
            \zxN{} \rar &\zxZ{\pi} \rar &\zxX{\pi} \rar& \zxN{}
        \end{ZX}
         & \equiv   \begin{ZX}
                        \zxN{} \rar &\zxZ{\pi} \rar & \zxN{}
                    \end{ZX} \circ \begin{ZX}
                                       \zxN{} \rar &\zxX{\pi} \rar & \zxN{}
                                   \end{ZX} \\
         & \equiv Z \circ X
        \\
         & = XZ                                                        \\
         & =
        \begin{pmatrix}
            0 & 1 \\
            1 & 0 \\
        \end{pmatrix}
        \begin{pmatrix}
            1 & 0  \\
            0 & -1 \\
        \end{pmatrix}
        \\
         & =
        \begin{pmatrix}
            0 & -1 \\
            1 & 0  \\
        \end{pmatrix}                                                 \\
         & = Y/i                                                       \\
         & \propto Y
    \end{align*}
    \caption{Pauli-Y gate represented as a ZX-Diagram}
\end{figure}

